\documentclass[]{elsarticle} %review=doublespace preprint=single 5p=2 column
%%% Begin My package additions %%%%%%%%%%%%%%%%%%%
\usepackage[hyphens]{url}

  \journal{BioR\(\chi\)iv} % Sets Journal name


\usepackage{lineno} % add

\usepackage{graphicx}
%%%%%%%%%%%%%%%% end my additions to header

\usepackage[T1]{fontenc}
\usepackage{lmodern}
\usepackage{amssymb,amsmath}
\usepackage{ifxetex,ifluatex}
\usepackage{fixltx2e} % provides \textsubscript
% use upquote if available, for straight quotes in verbatim environments
\IfFileExists{upquote.sty}{\usepackage{upquote}}{}
\ifnum 0\ifxetex 1\fi\ifluatex 1\fi=0 % if pdftex
  \usepackage[utf8]{inputenc}
\else % if luatex or xelatex
  \usepackage{fontspec}
  \ifxetex
    \usepackage{xltxtra,xunicode}
  \fi
  \defaultfontfeatures{Mapping=tex-text,Scale=MatchLowercase}
  \newcommand{\euro}{€}
\fi
% use microtype if available
\IfFileExists{microtype.sty}{\usepackage{microtype}}{}
\usepackage[margin=1in]{geometry}
\bibliographystyle{elsarticle-harv}
\ifxetex
  \usepackage[setpagesize=false, % page size defined by xetex
              unicode=false, % unicode breaks when used with xetex
              xetex]{hyperref}
\else
  \usepackage[unicode=true]{hyperref}
\fi
\hypersetup{breaklinks=true,
            bookmarks=true,
            pdfauthor={},
            pdftitle={Does subscapularis splitting technique during the Latarjet procedure matter? Muscle moment arms and lines of action in a cadaveric model},
            colorlinks=false,
            urlcolor=blue,
            linkcolor=magenta,
            pdfborder={0 0 0}}
\urlstyle{same}  % don't use monospace font for urls

\setcounter{secnumdepth}{5}
% Pandoc toggle for numbering sections (defaults to be off)

% Pandoc syntax highlighting
\usepackage{color}
\usepackage{fancyvrb}
\newcommand{\VerbBar}{|}
\newcommand{\VERB}{\Verb[commandchars=\\\{\}]}
\DefineVerbatimEnvironment{Highlighting}{Verbatim}{commandchars=\\\{\}}
% Add ',fontsize=\small' for more characters per line
\usepackage{framed}
\definecolor{shadecolor}{RGB}{248,248,248}
\newenvironment{Shaded}{\begin{snugshade}}{\end{snugshade}}
\newcommand{\AlertTok}[1]{\textcolor[rgb]{0.94,0.16,0.16}{#1}}
\newcommand{\AnnotationTok}[1]{\textcolor[rgb]{0.56,0.35,0.01}{\textbf{\textit{#1}}}}
\newcommand{\AttributeTok}[1]{\textcolor[rgb]{0.77,0.63,0.00}{#1}}
\newcommand{\BaseNTok}[1]{\textcolor[rgb]{0.00,0.00,0.81}{#1}}
\newcommand{\BuiltInTok}[1]{#1}
\newcommand{\CharTok}[1]{\textcolor[rgb]{0.31,0.60,0.02}{#1}}
\newcommand{\CommentTok}[1]{\textcolor[rgb]{0.56,0.35,0.01}{\textit{#1}}}
\newcommand{\CommentVarTok}[1]{\textcolor[rgb]{0.56,0.35,0.01}{\textbf{\textit{#1}}}}
\newcommand{\ConstantTok}[1]{\textcolor[rgb]{0.00,0.00,0.00}{#1}}
\newcommand{\ControlFlowTok}[1]{\textcolor[rgb]{0.13,0.29,0.53}{\textbf{#1}}}
\newcommand{\DataTypeTok}[1]{\textcolor[rgb]{0.13,0.29,0.53}{#1}}
\newcommand{\DecValTok}[1]{\textcolor[rgb]{0.00,0.00,0.81}{#1}}
\newcommand{\DocumentationTok}[1]{\textcolor[rgb]{0.56,0.35,0.01}{\textbf{\textit{#1}}}}
\newcommand{\ErrorTok}[1]{\textcolor[rgb]{0.64,0.00,0.00}{\textbf{#1}}}
\newcommand{\ExtensionTok}[1]{#1}
\newcommand{\FloatTok}[1]{\textcolor[rgb]{0.00,0.00,0.81}{#1}}
\newcommand{\FunctionTok}[1]{\textcolor[rgb]{0.00,0.00,0.00}{#1}}
\newcommand{\ImportTok}[1]{#1}
\newcommand{\InformationTok}[1]{\textcolor[rgb]{0.56,0.35,0.01}{\textbf{\textit{#1}}}}
\newcommand{\KeywordTok}[1]{\textcolor[rgb]{0.13,0.29,0.53}{\textbf{#1}}}
\newcommand{\NormalTok}[1]{#1}
\newcommand{\OperatorTok}[1]{\textcolor[rgb]{0.81,0.36,0.00}{\textbf{#1}}}
\newcommand{\OtherTok}[1]{\textcolor[rgb]{0.56,0.35,0.01}{#1}}
\newcommand{\PreprocessorTok}[1]{\textcolor[rgb]{0.56,0.35,0.01}{\textit{#1}}}
\newcommand{\RegionMarkerTok}[1]{#1}
\newcommand{\SpecialCharTok}[1]{\textcolor[rgb]{0.00,0.00,0.00}{#1}}
\newcommand{\SpecialStringTok}[1]{\textcolor[rgb]{0.31,0.60,0.02}{#1}}
\newcommand{\StringTok}[1]{\textcolor[rgb]{0.31,0.60,0.02}{#1}}
\newcommand{\VariableTok}[1]{\textcolor[rgb]{0.00,0.00,0.00}{#1}}
\newcommand{\VerbatimStringTok}[1]{\textcolor[rgb]{0.31,0.60,0.02}{#1}}
\newcommand{\WarningTok}[1]{\textcolor[rgb]{0.56,0.35,0.01}{\textbf{\textit{#1}}}}

% tightlist command for lists without linebreak
\providecommand{\tightlist}{%
  \setlength{\itemsep}{0pt}\setlength{\parskip}{0pt}}






\begin{document}


\begin{frontmatter}

  \title{Does subscapularis splitting technique during the Latarjet
procedure matter? Muscle moment arms and lines of action in a cadaveric
model}
    \author[Centre for Sports Research,Barwon Centre for Orthopaedic
Research and Education (B-CORE)]{Aaron S. Fox\corref{1}}
  
    \author[Barwon Centre for Orthopaedic Research and Education
(B-CORE),School of Medicine,Orthopaedic Department]{Richard S. Page}
  
    \author[TODO]{Janina Henze}
  
    \author[Department of Orthopaedic Surgery]{Lukas Ernstbrunner}
  
    \author[Department of Biomedical Engineering]{David C. Ackland}
  
      \address[Centre for Sports Research]{Centre for Sports Research,
School of Exercise and Nutrition Sciences, Deakin University, Geelong,
Australia}
    \address[Barwon Centre for Orthopaedic Research and Education
(B-CORE)]{Barwon Centre for Orthopaedic Research and Education (B-CORE),
Barwon Health, St John of Jod Hospital and Deakin University, Geelong,
Australia}
    \address[School of Medicine]{School of Medicine, Deakin University,
Geelong, Australia}
    \address[Orthopaedic Department]{Orthopaedic Department, University
Hospital Geelong, Barwon Health, Geelong, Australia}
    \address[Department of Orthopaedic Surgery]{Department of
Orthopaedic Surgery, University Hospital Balgrist, Zurich, Switzerland}
    \address[Department of Biomedical Engineering]{Department of
Biomedical Engineering, University of Melbourne, Parkville, Australia}
      \cortext[1]{Corresponding Author: aaron.f@deakin.edu.au}
  
  \begin{abstract}
  TODO: Add abstract\ldots{}
  \end{abstract}
  
 \end{frontmatter}

\hypertarget{introduction}{%
\section{Introduction}\label{introduction}}

\textbf{\emph{TODO: copy across a lot of this to multiple split
paper\ldots{}}}

Shoulder instability injuries occur with excessive force that translates
the humeral head out of the glenohumeral joint socket
(\textbf{Thangarajah2016?}), and are a concerning problem affecting
young athletes in overhead collision sports (e.g.~Australian football,
rugby) (\textbf{Orchard2013?}). Effective clinical care is vital to
avoid recurrent injuries, as well as reduced shoulder function and joint
degradation (\textbf{Thangarajah2016?}). Surgery is often needed to
address pathology, restore function and correct stability
(\textbf{Kavaja2012?}). The Latarjet procedure is a non-anatomic, open
shoulder reconstruction involving a bone block via transfer of the
coracoid process to the anterior glenoid with the attached conjoint
tendon (\textbf{Latarjet1954?}). The Latarjet procedure is commonly used
in cases with significant glenoid bone loss, large humerus compression
fractures, or glenoid and humeral bone defects (\textbf{Millett2005?})
--- and is effective in combatting recurrent anterior instability injury
(\textbf{Bessiere2014?}). Latarjet procedures are also emerging as the
preferred option for surgical shoulder stabilisation, especially in
contact sport settings (\textbf{Bonazza2017?}).

The Latarjet procedure requires high precision, and subtle variations in
surgical technique may impact the likelihood of degenerative changes and
subsequent injury (\textbf{Ghodadra2010?}). An important choice in the
Latarjet procedure is the treatment of the subscapularis muscle
(\textbf{Bhatia2014?}). The subscapularis muscle must be manipulated to
access the anterior portion of the glenohumeral joint and place the
coracoid bone graft. Early iterations of the Latarjet procedure
completely dissected the subscapularis tendon and reflected the muscle
to expose the anterior joint capsule. Complete release of the
subscapularis during surgery can elevate the risk of future tears to the
tendon (\textbf{Lazarus2000?}). Further, the subscapularis is a strong
anterior stabiliser of the joint (\textbf{Lee2000?}) and therefore
maintaining its integrity is relevant to joint stability. Subsequently,
the current recommended approach (\textbf{Bhatia2014?}) is to use a
horizontal split through the muscle fibres (i.e.~a subscapularis split).
Using the subscapularis split approach, the final position of the
conjoint tendon passes through the split, adding tension to the inferior
portion in extreme degrees of abduction and/or external rotation
(\textbf{Yamamoto2013?}). This added tension in the subscapularis
enhances anterior and inferior joint stability (\textbf{Yamamoto2013?}).

Although a subscapularis splitting approach for the Latarjet procedure
is advocated, variable techniques for the location of the split are
reported (\textbf{Shah2012?}). This includes splitting the subscapularis
at: (i) the junction of its superior and middle thirds (i.e.~one-third
from the top) (\textbf{Burkhart2007?}); (ii) the junction of its upper
two-thirds and lower one-third (i.e.~one-third from the bottom)
(\textbf{Shah2012?}); or (iii) along the fibres at the `middle third'
(i.e.~mid-point of the muscle) (\textbf{Allain1998?}). Understanding
which split location optimises stability and function is key for
providing further guidance around the Latarjet procedure. However, the
effect on stability and muscle function for varying split locations has
not been extensively tested. Of those studies which have investigated
the stabilising mechanisms of the Latarjet procedure
(\textbf{Wellmann2012?}) --- the majority have used a split at the upper
two and lower one-thirds of the muscle (\textbf{Wellmann2012?}); while
one did not report the location (\textbf{Giles2013?}). There is a clear
gap in understanding how variable subscapularis split locations furing
the Latarjet procedure impacts its mechanical stabilising effect.
Further, none of the aforemention mechanistic studies
(\textbf{Wellmann2012?}) have examined muscle anatomy and function. The
Latarjet procedure and splitting technique used will have a substantial
effect on subscapularis muscle moment arms and lines of action. The
moment arm of a muscle largely determines its role as a stabiliser and
prime mover (\textbf{Ackland2008?}). A muscle's line of action dictates
the direction in which it produces force (\textbf{Ackland2009?}).
Muscles whose line of action produces predominantly compressive forces
stabilise the shoulder, while those that produce predominantly shear
forces may help cause instability (\textbf{Ackland2003?}). Understanding
how subscapularis splitting alters the moment arms and lines of action
of the subscapularis can reveal further information about this muscles
contribution to shoulder joint stability folowing the Latarjet
procedure. Assessing the various splitting techniques may also provide
support for an optimal method.

This study examines how varying subscapularis splitting techniques
within the Latarjet procedure impacts subscapularis muscle function and
the potential for shoulder joint (in)stability. First, the moment arms
and lines of action of subscapularis muscle sub-regions across various
glenohumeral joint positions under different subscapularis splitting
techniques are examined. Second, these data are used to assess the
potential contributions of the subscapularis muscle sub-regions to
glenohumeral joint stability under the different subscapularis splitting
techniques.

\hypertarget{methods}{%
\section{Methods}\label{methods}}

\hypertarget{specimen-preparation}{%
\subsection{Specimen Preparation}\label{specimen-preparation}}

\textbf{\emph{XXXXX}} (\textbf{\emph{X male, X female; \ldots enter
participant details}}) fresh-frozen, entire upper extremities were
obtained from human cadavera. Ethics approval for the use of specimen in
this study was obtained from the \textbf{\emph{Health Sciences Human
Ethics Sub-Committee, University of Melbourne}}. All specimens were
arthroscopically screened to ensure they were free of degenerative
changes such as osteoarthritis, rotator cuff tears and significant joint
contracture. Specimens were thawed at room temperature 24 hours prior to
dissecting and testing.

\textbf{TODO: add details about specimen preparation\ldots{}}

\textbf{TODO: include details from Janina re: Latarjet
procedures\ldots{}}

\hypertarget{experimental-protocol}{%
\subsection{Experimental Protocol}\label{experimental-protocol}}

\textbf{\ldots{}}

Nylon lines were attached to the suture of each proximal tendon of the
subscapularis regions and passed through a perforate plate to a free
hanging weight of \textbf{\emph{\ldots insert load\ldots{}}}. This
maintained muscle-tendon unit tension while minimising load induced
muscle-tendon lengthening, and assisted in producing joint congruency
during testing. Each tendon unit was pulled toward the centroid of its
proximal origin, thus reproducing the approximate line of action of the
muscle-tendon region.

The moment arms and line of action for each sub-region of the
subscapularis muscle were calculated across a series of static
positions. The humerus was passively held at \textbf{\emph{\ldots insert
test angles\ldots{}}} of elevation in the scapular plane, as well as at
\textbf{\emph{\ldots insert abduction external rotation
positions\ldots{}}}.

Specimens were radiographed using X-ray fluoroscopy from
\textbf{\emph{\ldots list the two directions\ldots{}}} at each joint
position (Fluoroscan InSight 2, Hologic Inc., Bedford, MA), and each
muscle subregions line of action was calculated from its wire
orientation with respect to the glenoid plane. A muscle subregions line
of action was defined from the directional cosines of the vector formed
between the most proximal tendon wrapping `via point' (i.e.~where the
tendon loses contact with the hmerus) and the centroid of the tendon
origin (\textbf{Ackland2019?}).

\textbf{TODO: muscle moment arm calculations\ldots{}}

For each muscle subregion, average stability ratios were computed to
assess the muscle's potential contributions to anterior/posterior and
superior/inferior glenhumeral joint stability across the tested
positions. Average anterior and superior stability ratios were
calculated by dividing the average anterior/posterior and
superior/inferior shear components of a muscle subregions line of
action, respectively, by the average magnitude of its compressive
component:

\[R_{A} = \frac{f_{y}}{|f_{x}|}; R_{S} = \frac{f_{z}}{|f_{x}|}\] where
\(R_{A}\) and \(R_{S}\) are the anterior and superior stability ratios,
respectively; \(f_{x}\), \(f_{y}\) and \(f_{z}\) are the directional
cosines of the vectors used to calculate the line of action in the
scapular reference frame (\textbf{ACkland2009?}). Where a muscle
subregions stability ratio was greater than one it was considered as
having destabilising potential, as the shear component of the line of
action was larger than the compressive component
(\textbf{Ackland2009?}). Conversely, awhere muscle subregions stability
ratio was less than one it was considered as having stabilising
potential, as the shear component of its line of action was smaller than
the compressive component (\textbf{Ackland2009?}). A positive versus
negative anterior stability ratio represented a muscle subregion with an
anterior versus posterior shear component, respectively; while a
positive versus negative superior stability ratio represented a muscle
subregion with a superior versus inferior shear component, respectively
(\textbf{Ackland2009?}).

Calculations were made for each specimen across the
\textbf{\emph{three}} subscapularis muscle split configurations.

We used a mixed effects linear model to examine changes in\ldots line of
action, moment arms\ldots fixed effects of subscapularis muscle region,
arm position and load, random effect of specimen\ldots{}

\hypertarget{results}{%
\section{Results}\label{results}}

\textbf{TODO: add results}

\begin{verbatim}
## # A tibble: 96 x 7
##    region position load  plane variable     statistic     p
##    <fct>  <fct>    <fct> <fct> <chr>            <dbl> <dbl>
##  1 ss1    abd0     0N    SP    lineOfAction     0.924 0.463
##  2 ss1    abd0     20N   SP    lineOfAction     0.929 0.511
##  3 ss1    abd0     40N   SP    lineOfAction     0.932 0.534
##  4 ss1    abd90    0N    SP    lineOfAction     0.924 0.461
##  5 ss1    abd90    20N   SP    lineOfAction     0.913 0.376
##  6 ss1    abd90    40N   SP    lineOfAction     0.900 0.290
##  7 ss1    ABER     0N    SP    lineOfAction     0.960 0.813
##  8 ss1    ABER     20N   SP    lineOfAction     0.934 0.555
##  9 ss1    ABER     40N   SP    lineOfAction     0.931 0.526
## 10 ss1    APP      0N    SP    lineOfAction     0.923 0.451
## # ... with 86 more rows
\end{verbatim}

\hypertarget{lines-of-action}{%
\subsection{Lines of Action}\label{lines-of-action}}

\hypertarget{scapular-plane}{%
\subsubsection{Scapular Plane}\label{scapular-plane}}

Three-way repeated measures ANOVA found statistically significant
subscapularis region x arm position () and subscapularis region x
conjoined tendon load () interaction effects; and statistically
significant subscapularis region() and arm position () main effects on
muscle line of action in the scapular plane. The main effect of
conjoined tendon load (), and interaction effects of arm position x
conjoined tendon load () and subscapularis region x arm position x
conjoined tendon load () on line of action in the scapular plane did not
reach statistical significance.\\
Post-hoc analysis of subscapularis region x arm position effects found a
statistically significant effect of arm position on the upper (),
lower-middle () and lower () regions of the muscle. No statistically
significant post-hoc effects (adjusted \emph{p} \textgreater{} 0.05)
were observed for conjoined tendon load on subscapularis region.

\hypertarget{discussion}{%
\section{Discussion}\label{discussion}}

\textbf{TODO: add discussion}

\hypertarget{conclusions}{%
\section{Conclusions}\label{conclusions}}

\textbf{TODO: add conclusions}

\begin{Shaded}
\begin{Highlighting}[]

\CommentTok{\#Basic print command}
\BuiltInTok{print}\NormalTok{(}\StringTok{\textquotesingle{}Using python in R!\textquotesingle{}}\NormalTok{)}

\CommentTok{\#Import pandas}
\end{Highlighting}
\end{Shaded}

\begin{verbatim}
## Using python in R!
\end{verbatim}

\begin{Shaded}
\begin{Highlighting}[]
\ImportTok{import}\NormalTok{ pandas }\ImportTok{as}\NormalTok{ pd}

\CommentTok{\#Import seaborn and load dataset}
\ImportTok{import}\NormalTok{ seaborn }\ImportTok{as}\NormalTok{ sns}
\NormalTok{iris }\OperatorTok{=}\NormalTok{ sns.load\_dataset(}\StringTok{\textquotesingle{}iris\textquotesingle{}}\NormalTok{)}
\NormalTok{iris.head()}
\end{Highlighting}
\end{Shaded}

\begin{verbatim}
##    sepal_length  sepal_width  petal_length  petal_width species
## 0           5.1          3.5           1.4          0.2  setosa
## 1           4.9          3.0           1.4          0.2  setosa
## 2           4.7          3.2           1.3          0.2  setosa
## 3           4.6          3.1           1.5          0.2  setosa
## 4           5.0          3.6           1.4          0.2  setosa
\end{verbatim}

\hypertarget{references}{%
\section*{References}\label{references}}
\addcontentsline{toc}{section}{References}


\end{document}
